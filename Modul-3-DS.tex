% Options for packages loaded elsewhere
\PassOptionsToPackage{unicode}{hyperref}
\PassOptionsToPackage{hyphens}{url}
%
\documentclass[
]{article}
\usepackage{lmodern}
\usepackage{amssymb,amsmath}
\usepackage{ifxetex,ifluatex}
\ifnum 0\ifxetex 1\fi\ifluatex 1\fi=0 % if pdftex
  \usepackage[T1]{fontenc}
  \usepackage[utf8]{inputenc}
  \usepackage{textcomp} % provide euro and other symbols
\else % if luatex or xetex
  \usepackage{unicode-math}
  \defaultfontfeatures{Scale=MatchLowercase}
  \defaultfontfeatures[\rmfamily]{Ligatures=TeX,Scale=1}
\fi
% Use upquote if available, for straight quotes in verbatim environments
\IfFileExists{upquote.sty}{\usepackage{upquote}}{}
\IfFileExists{microtype.sty}{% use microtype if available
  \usepackage[]{microtype}
  \UseMicrotypeSet[protrusion]{basicmath} % disable protrusion for tt fonts
}{}
\makeatletter
\@ifundefined{KOMAClassName}{% if non-KOMA class
  \IfFileExists{parskip.sty}{%
    \usepackage{parskip}
  }{% else
    \setlength{\parindent}{0pt}
    \setlength{\parskip}{6pt plus 2pt minus 1pt}}
}{% if KOMA class
  \KOMAoptions{parskip=half}}
\makeatother
\usepackage{xcolor}
\IfFileExists{xurl.sty}{\usepackage{xurl}}{} % add URL line breaks if available
\IfFileExists{bookmark.sty}{\usepackage{bookmark}}{\usepackage{hyperref}}
\hypersetup{
  pdftitle={Modul 3 DS},
  pdfauthor={Rafly Pradana Putra},
  hidelinks,
  pdfcreator={LaTeX via pandoc}}
\urlstyle{same} % disable monospaced font for URLs
\usepackage[margin=1in]{geometry}
\usepackage{color}
\usepackage{fancyvrb}
\newcommand{\VerbBar}{|}
\newcommand{\VERB}{\Verb[commandchars=\\\{\}]}
\DefineVerbatimEnvironment{Highlighting}{Verbatim}{commandchars=\\\{\}}
% Add ',fontsize=\small' for more characters per line
\usepackage{framed}
\definecolor{shadecolor}{RGB}{248,248,248}
\newenvironment{Shaded}{\begin{snugshade}}{\end{snugshade}}
\newcommand{\AlertTok}[1]{\textcolor[rgb]{0.94,0.16,0.16}{#1}}
\newcommand{\AnnotationTok}[1]{\textcolor[rgb]{0.56,0.35,0.01}{\textbf{\textit{#1}}}}
\newcommand{\AttributeTok}[1]{\textcolor[rgb]{0.77,0.63,0.00}{#1}}
\newcommand{\BaseNTok}[1]{\textcolor[rgb]{0.00,0.00,0.81}{#1}}
\newcommand{\BuiltInTok}[1]{#1}
\newcommand{\CharTok}[1]{\textcolor[rgb]{0.31,0.60,0.02}{#1}}
\newcommand{\CommentTok}[1]{\textcolor[rgb]{0.56,0.35,0.01}{\textit{#1}}}
\newcommand{\CommentVarTok}[1]{\textcolor[rgb]{0.56,0.35,0.01}{\textbf{\textit{#1}}}}
\newcommand{\ConstantTok}[1]{\textcolor[rgb]{0.00,0.00,0.00}{#1}}
\newcommand{\ControlFlowTok}[1]{\textcolor[rgb]{0.13,0.29,0.53}{\textbf{#1}}}
\newcommand{\DataTypeTok}[1]{\textcolor[rgb]{0.13,0.29,0.53}{#1}}
\newcommand{\DecValTok}[1]{\textcolor[rgb]{0.00,0.00,0.81}{#1}}
\newcommand{\DocumentationTok}[1]{\textcolor[rgb]{0.56,0.35,0.01}{\textbf{\textit{#1}}}}
\newcommand{\ErrorTok}[1]{\textcolor[rgb]{0.64,0.00,0.00}{\textbf{#1}}}
\newcommand{\ExtensionTok}[1]{#1}
\newcommand{\FloatTok}[1]{\textcolor[rgb]{0.00,0.00,0.81}{#1}}
\newcommand{\FunctionTok}[1]{\textcolor[rgb]{0.00,0.00,0.00}{#1}}
\newcommand{\ImportTok}[1]{#1}
\newcommand{\InformationTok}[1]{\textcolor[rgb]{0.56,0.35,0.01}{\textbf{\textit{#1}}}}
\newcommand{\KeywordTok}[1]{\textcolor[rgb]{0.13,0.29,0.53}{\textbf{#1}}}
\newcommand{\NormalTok}[1]{#1}
\newcommand{\OperatorTok}[1]{\textcolor[rgb]{0.81,0.36,0.00}{\textbf{#1}}}
\newcommand{\OtherTok}[1]{\textcolor[rgb]{0.56,0.35,0.01}{#1}}
\newcommand{\PreprocessorTok}[1]{\textcolor[rgb]{0.56,0.35,0.01}{\textit{#1}}}
\newcommand{\RegionMarkerTok}[1]{#1}
\newcommand{\SpecialCharTok}[1]{\textcolor[rgb]{0.00,0.00,0.00}{#1}}
\newcommand{\SpecialStringTok}[1]{\textcolor[rgb]{0.31,0.60,0.02}{#1}}
\newcommand{\StringTok}[1]{\textcolor[rgb]{0.31,0.60,0.02}{#1}}
\newcommand{\VariableTok}[1]{\textcolor[rgb]{0.00,0.00,0.00}{#1}}
\newcommand{\VerbatimStringTok}[1]{\textcolor[rgb]{0.31,0.60,0.02}{#1}}
\newcommand{\WarningTok}[1]{\textcolor[rgb]{0.56,0.35,0.01}{\textbf{\textit{#1}}}}
\usepackage{graphicx,grffile}
\makeatletter
\def\maxwidth{\ifdim\Gin@nat@width>\linewidth\linewidth\else\Gin@nat@width\fi}
\def\maxheight{\ifdim\Gin@nat@height>\textheight\textheight\else\Gin@nat@height\fi}
\makeatother
% Scale images if necessary, so that they will not overflow the page
% margins by default, and it is still possible to overwrite the defaults
% using explicit options in \includegraphics[width, height, ...]{}
\setkeys{Gin}{width=\maxwidth,height=\maxheight,keepaspectratio}
% Set default figure placement to htbp
\makeatletter
\def\fps@figure{htbp}
\makeatother
\setlength{\emergencystretch}{3em} % prevent overfull lines
\providecommand{\tightlist}{%
  \setlength{\itemsep}{0pt}\setlength{\parskip}{0pt}}
\setcounter{secnumdepth}{-\maxdimen} % remove section numbering

\title{Modul 3 DS}
\author{Rafly Pradana Putra}
\date{12/2/2020}

\begin{document}
\maketitle

\hypertarget{dasar-teori}{%
\subsection{Dasar Teori}\label{dasar-teori}}

Variasi tipe data pada R memfasilitasi keberagaman jenis variabel data.
Sebagai contoh, terdapat data yang terdiri dari sekumpulan angka dan
data lain yang berisi sekumpulan karakter. Pada contoh lain, ada pula
data yang berbentuk tabel maupun kumpulan (list) angka sederhana. Dengan
bantuan fungsi class, kita akan mendapatkan kemudahan dalam
mendefinisikan tipe data yang kita miliki:

\begin{Shaded}
\begin{Highlighting}[]
\NormalTok{a <-}\StringTok{ }\DecValTok{2}
\KeywordTok{class}\NormalTok{(a)}
\end{Highlighting}
\end{Shaded}

\begin{verbatim}
## [1] "numeric"
\end{verbatim}

Agar dapat bekerja secara efisien dalam menggunakan bahasa pemrograman
R, penting untuk mempelajari terlebih dahulu tipe data dari
variabel-variabel yang kita miliki sehingga akan mempermudah dalam
penentuan proses analisis data yang dapat dilakukan terhadap
variabelvariabel tersebut.

\hypertarget{data-frames}{%
\subsection{Data Frames}\label{data-frames}}

Cara paling umum yang dapat digunakan untuk menyimpan dataset dalam R
adalah dalam tipe data frame. Secara konseptual, kita dapat menganggap
data frame sebagai tabel yang terdiri dari baris yang memiliki nilai
pengamatan dan berbagai variabel yang didefinisikan dalam bentuk kolom.
Tipe data ini sangat umum digunakan untuk dataset, karena data frame
dapat menggabungkan berbagai jenis tipe data dalam satu objek. Untuk
memahami tipe data frame, silahkan mengakses contoh dataset pada
library(dslabs) dan pilih dataset ``murders'' menggunakan fungsi data:

\begin{Shaded}
\begin{Highlighting}[]
\KeywordTok{library}\NormalTok{(dslabs)}
\KeywordTok{data}\NormalTok{(murders)}
\end{Highlighting}
\end{Shaded}

Untuk memastikan bahwa dataset tersebut tipenya adalah data frame, dapat
digunakan perintah berikut:

\begin{Shaded}
\begin{Highlighting}[]
\KeywordTok{class}\NormalTok{(murders)}
\end{Highlighting}
\end{Shaded}

\begin{verbatim}
## [1] "data.frame"
\end{verbatim}

Untuk memeriksa lebih lanjut isi dataset, dapat pula digunakan fungsi
str untuk mencari tahu lebih rinci mengenai struktur suatu objek:

\begin{Shaded}
\begin{Highlighting}[]
\KeywordTok{str}\NormalTok{(murders)}
\end{Highlighting}
\end{Shaded}

\begin{verbatim}
## 'data.frame':    51 obs. of  5 variables:
##  $ state     : chr  "Alabama" "Alaska" "Arizona" "Arkansas" ...
##  $ abb       : chr  "AL" "AK" "AZ" "AR" ...
##  $ region    : Factor w/ 4 levels "Northeast","South",..: 2 4 4 2 4 4 1 2 2 2 ...
##  $ population: num  4779736 710231 6392017 2915918 37253956 ...
##  $ total     : num  135 19 232 93 1257 ...
\end{verbatim}

Dengan menggunakan fungsi str, dapat diketahui bahwa dataset ``murders''
terdiri dari 51 baris dan lima variabel: state, abb, region, population,
dan total. Selanjutnya, untuk melihat contoh enam baris pertama pada
dataset, dapat digunakan fungsi head:

\begin{Shaded}
\begin{Highlighting}[]
\KeywordTok{head}\NormalTok{(murders)}
\end{Highlighting}
\end{Shaded}

\begin{verbatim}
##        state abb region population total
## 1    Alabama  AL  South    4779736   135
## 2     Alaska  AK   West     710231    19
## 3    Arizona  AZ   West    6392017   232
## 4   Arkansas  AR  South    2915918    93
## 5 California  CA   West   37253956  1257
## 6   Colorado  CO   West    5029196    65
\end{verbatim}

Untuk analisis awal tiap variabel yang diwakili dalam bentuk kolom pada
tipe data frame, dapat digunakan operator aksesor (\$) dengan cara
berikut:

\begin{Shaded}
\begin{Highlighting}[]
\NormalTok{murders}\OperatorTok{$}\NormalTok{population}
\end{Highlighting}
\end{Shaded}

\begin{verbatim}
##  [1]  4779736   710231  6392017  2915918 37253956  5029196  3574097   897934
##  [9]   601723 19687653  9920000  1360301  1567582 12830632  6483802  3046355
## [17]  2853118  4339367  4533372  1328361  5773552  6547629  9883640  5303925
## [25]  2967297  5988927   989415  1826341  2700551  1316470  8791894  2059179
## [33] 19378102  9535483   672591 11536504  3751351  3831074 12702379  1052567
## [41]  4625364   814180  6346105 25145561  2763885   625741  8001024  6724540
## [49]  1852994  5686986   563626
\end{verbatim}

Untuk mengetahui nama-nama dari lima variabel yang dapat dievaluasi
menggunakan operator aksesor, sebelumnya, melalui fungsi str, telah kita
ketahui bahwa variabel yang dimiliki dataset adalah: state, abb, region,
population, dan total. Sebagai alternatif, terdapat pula fungsi name,
yang dapat digunakan seperti contoh dibawah ini: Note that the
\texttt{echo\ =\ FALSE} parameter was added to the code chunk to prevent
printing of the R code that generated the plot.

\begin{Shaded}
\begin{Highlighting}[]
\KeywordTok{names}\NormalTok{(murders)}
\end{Highlighting}
\end{Shaded}

\begin{verbatim}
## [1] "state"      "abb"        "region"     "population" "total"
\end{verbatim}

\end{document}
